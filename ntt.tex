\documentclass{beamer}	% Compile at least twice!
%\setbeamertemplate{navigation symbols}{}
\usetheme{Warsaw}
%\useinnertheme{rectangles}
%\useoutertheme{infolines}
\useoutertheme[title,section,subsection=true]{smoothbars}
 
% -------------------
% Packages
% -------------------
\usepackage{
	amsmath,			% Math Environments
	amssymb,			% Extended Symbols
	enumerate,		    % Enumerate Environments
	graphicx,			% Include Images
	lastpage,			% Reference Lastpage
	multicol,			% Use Multi-columns
	multirow,			% Use Multi-rows
	pifont,			    % For Checkmarks
	stmaryrd			% For brackets
}
\usepackage[english]{babel}


% -------------------
% Colors
% -------------------
\definecolor{UniOrange}{RGB}{212,69,0}
\definecolor{UniGray}{RGB}{62,61,60}
%\definecolor{UniRed}{HTML}{B31B1B}
%\definecolor{UniGray}{HTML}{222222}
\setbeamercolor{title}{fg=UniGray}
\setbeamercolor{frametitle}{fg=UniOrange}
\setbeamercolor{structure}{fg=UniOrange}
\setbeamercolor{section in head/foot}{bg=UniGray}
\setbeamercolor{author in head/foot}{bg=UniGray}
\setbeamercolor{date in head/foot}{fg=UniGray}
\setbeamercolor{structure}{fg=UniOrange}
\setbeamercolor{local structure}{fg=black}
\beamersetuncovermixins{\opaqueness<1>{0}}{\opaqueness<2->{15}}


% -------------------
% Fonts & Layout
% -------------------
\useinnertheme{default}
\usefonttheme{serif}
\usepackage{palatino}
\setbeamerfont{title like}{shape=\scshape}
\setbeamerfont{frametitle}{shape=\scshape}
\setbeamertemplate{itemize items}[circle]
%\setbeamertemplate{enumerate items}[default]


% -------------------
% Commands
% -------------------

% Special Characters
\newcommand{\N}{\mathbb{N}}
\newcommand{\Z}{\mathbb{Z}}
\newcommand{\Q}{\mathbb{Q}}
\newcommand{\R}{\mathbb{R}}
\newcommand{\C}{\mathbb{C}}

% Math Operators
\DeclareMathOperator{\im}{im}
\DeclareMathOperator{\Span}{span}

% Special Commands
\newcommand{\pf}{\noindent\emph{Proof. }}
\newcommand{\ds}{\displaystyle}
\newcommand{\defeq}{\stackrel{\text{def}}{=}}
\newcommand{\ov}[1]{\overline{#1}}
\newcommand{\ma}[1]{\stackrel{#1}{\longrightarrow}}
\newcommand{\twomatrix}[4]{\begin{pmatrix} #1 & #2 \\ #3 & #4 \end{pmatrix}}


% -------------------
% Tikz & PGF
% -------------------
\usepackage{tikz}
\usepackage{tikz-cd}
\usetikzlibrary{
	calc,
	decorations.pathmorphing,
	matrix,arrows,
	positioning,
	shapes.geometric
}
\usepackage{pgfplots}
\pgfplotsset{compat=newest}


% -------------------
% Theorem Environments
% -------------------
\theoremstyle{plain}
\newtheorem{thm}{Theorem}[section]
\newtheorem{prop}{Proposition}[section]
\newtheorem{lem}{Lemma}[section]
\newtheorem{cor}{Corollary}[section]
\theoremstyle{definition}
\newtheorem{ex}{Example}[section]
\newtheorem{nex}{Non-Example}[section]
\newtheorem{dfn}{Definition}[section]
\theoremstyle{remark}
\newtheorem{rem}{Remark}[section] 
\numberwithin{equation}{section}


% -------------------
% Title Page
% -------------------
\title{\textcolor{white}{Notes on NTT}}
\subtitle{\textcolor{white}{}}  
\author{A.S.}
\date{April 1, 2020} 


% -------------------
% Content
% -------------------
\begin{document}

\begin{frame}
For simplicity, let $n = q - 1$.\\
Observe that, when $k \ne 0$, 
\[
\sum_{i=0}^{n-1}\alpha^{i \cdot k} = 0, \forall \alpha \in \Z_q^*,
\]
and when $k = 0$, obviously
\[
\sum_{i=0}^{n-1}\alpha^0 = n = -1 \in \mathbb{Z}_q, \forall \alpha \in \Z_q^*.
\]
\end{frame}

\begin{frame}
Therefore we have
\begin{align*}
f(i) & = -\sum_{k,j}{f(j) \cdot \alpha^{k \cdot (j-i)}} \\
     & = -\sum_{k,j}{f(j) \cdot \alpha^{k \cdot j - k \cdot i}} \\
     & = -\sum_{k,j}{f(j) \cdot \alpha^{k \cdot j} \cdot \alpha^{-i \cdot k}} \\
     & = -\sum_k{F(k) \cdot \alpha^{-i \cdot k}},
\end{align*}
where $F(k) \overset{def}{=} \sum_{j}{f(j) \cdot \alpha^{k \cdot j}}$.
\end{frame}

\begin{frame}
We can compute $F(k)$ recursively ---
\begin{align*}
F^{(0)}(k) &= \sum_{j=0}^{n-1}{f(j) \cdot \alpha^{k \cdot j}} \\
           &= \sum_{l=0}^{\frac{n}{2}-1}{f(2l) \cdot \alpha^{k \cdot 2l}} + \alpha^{k} \cdot \sum_{l=0}^{\frac{n}{2}-1}{f(2l+1) \cdot \alpha^{k \cdot 2l}} \\
           &= F^{(1)}_0(k \bmod \frac{n}{2}) + \alpha^{k} \cdot F^{(1)}_1(k \bmod \frac{n}{2}),
\end{align*}
where $F^{(1)}_0$ and $F^{(1)}_1$ are the NTT's of with halved parameter $n$.

\end{frame}

\begin{frame}
We can compute $F(k)$ recursively ---
\begin{align*}
F^{(0)}(k) &= \sum_{j=0}^{n-1}{f(j) \cdot \alpha^{k \cdot j}} \\
&= \sum_{l=0}^{\frac{n}{2}-1}{f(2l) \cdot \alpha^{k \cdot 2l}} + \alpha^{k} \cdot \sum_{l=0}^{\frac{n}{2}-1}{f(2l+1) \cdot \alpha^{k \cdot 2l}} \\
&= F^{(1)}_0(k \bmod \frac{n}{2}) + \alpha^{k} \cdot F^{(1)}_1(k \bmod \frac{n}{2}).
\end{align*}
Notice the computation of $F^{(1)}_0$ uses $f(j)$'s for \textcolor{red}{even} $j$'s, and that of $F^{(1)}_1$ uses $f(j)$'s for \textcolor{red}{odd} $j$'s.

\end{frame}

\begin{frame}
We can compute $F(k)$ recursively ---
\begin{align*}
F^{(0)}(k) &= F^{(1)}_0(k') + \alpha^{k} \cdot F^{(1)}_1(k') \\
           &= F^{(2)}_0(k' \bmod \frac{n}{4}) + \alpha^{k'} \cdot F^{(2)}_1(k' \bmod \frac{n}{4})  \\
           &\phantom{=} + \alpha^{k} \cdot F^{(2)}_2(k' \bmod \frac{n}{4}) + \alpha^{k+k'} \cdot F^{(2)}_3(k' \bmod \frac{n}{4}) \\
           &= \cdots,
\end{align*}
where $k' = k \bmod \frac{n}{2}$.
\end{frame}

\begin{frame}
We can compute $F(k)$ recursively ---
\begin{align*}
F^{(0)}(k) &= F^{(1)}_0(k') + \alpha^{k} \cdot F^{(1)}_1(k') \\
&= F^{(2)}_0(k' \bmod \frac{n}{4}) + \alpha^{k'} \cdot F^{(2)}_1(k' \bmod \frac{n}{4})  \\
&\phantom{=} + \alpha^{k} \cdot F^{(2)}_2(k' \bmod \frac{n}{4}) + \alpha^{k+k'} \cdot F^{(2)}_3(k' \bmod \frac{n}{4}) \\
&= \cdots.
\end{align*}
The computation of $F^{(2)}_0, F^{(2)}_1$ uses $f(j)$'s for \textcolor{red}{even} $j$'s, and that of $F^{(2)}_2, F^{(2)}_3$ uses $f(j)$'s for \textcolor{red}{odd} $j$'s.
\end{frame}

\begin{frame}
We can compute $F(k)$ recursively ---
\begin{align*}
F^{(0)}(k) &= F^{(1)}_0(k') + \alpha^{k} \cdot F^{(1)}_1(k') \\
&= F^{(2)}_0(k' \bmod \frac{n}{4}) + \alpha^{k'} \cdot F^{(2)}_1(k' \bmod \frac{n}{4})  \\
&\phantom{=} + \alpha^{k} \cdot F^{(2)}_2(k' \bmod \frac{n}{4}) + \alpha^{k+k'} \cdot F^{(2)}_3(k' \bmod \frac{n}{4}) \\
&= \cdots.
\end{align*}
The computation of $F^{(2)}_0, F^{(2)}_1$ uses $f(j)$'s for \textcolor{red}{even} $j$'s, and that of $F^{(2)}_2, F^{(2)}_3$ uses $f(j)$'s for \textcolor{red}{odd} $j$'s. The computation of $F^{(2)}_0, F^{(2)}_2$ uses $f(j)$'s for the $j$'s s.t. \textcolor{red}{$(j \bmod 2) \bmod 4 = 0$}, and that of $F^{(2)}_1, F^{(2)}_3$ uses $f(j)$'s for the $j$'s s.t. \textcolor{red}{$(j \bmod 2) \bmod 4 \ne 0$}.
\end{frame}

\begin{frame}
We say an index is odder than another if it has $1$ in a lower bits. We place the $j$'s following the setting (which is the origin of bits-flipping):

\phantom{sep} 

% \phantom{The more even --- }the upper.
{\bf
The more even --- the upper;

%\phantom{sep}

The odder --- the lower.}

%\phantom{The more even --- }the lower.

\phantom{sep}

Hence the butterfly structure ---
\end{frame}

\begin{frame}
\begin{figure}
\begin{tikzpicture}
\draw[line width=3] (1,1) -- (11,1);
\draw[line width=3] (1,2) -- (11,2);
\draw[line width=3] (1,3) -- (11,3);
\draw[line width=3] (1,4) -- (11,4);
\draw[line width=3] (1,5) -- (11,5);
\draw[line width=3] (1,6) -- (11,6);
\draw[line width=3] (1,7) -- (11,7);
\draw[line width=3] (1,8) -- (11,8);

\draw[line width=3] (8,1) -- (10,5);
\draw[line width=3] (8,2) -- (10,6);
\draw[line width=3] (8,3) -- (10,7);
\draw[line width=3] (8,4) -- (10,8);

\draw[line width=3] (10,1) -- (8,5);
\draw[line width=3] (10,2) -- (8,6);
\draw[line width=3] (10,3) -- (8,7);
\draw[line width=3] (10,4) -- (8,8);

\draw[line width=3] (5,1) -- (7,3);
\draw[line width=3] (5,2) -- (7,4);

\draw[line width=3] (7,1) -- (5,3);
\draw[line width=3] (7,2) -- (5,4);

\draw[line width=3] (5,5) -- (7,7);
\draw[line width=3] (5,6) -- (7,8);

\draw[line width=3] (7,5) -- (5,7);
\draw[line width=3] (7,6) -- (5,8);

\draw[line width=3] (2,1) -- (4,2);

\draw[line width=3] (4,1) -- (2,2);

\draw[line width=3] (2,3) -- (4,4);

\draw[line width=3] (4,3) -- (2,4);

\draw[line width=3] (2,5) -- (4,6);

\draw[line width=3] (4,5) -- (2,6);

\draw[line width=3] (2,7) -- (4,8);

\draw[line width=3] (4,7) -- (2,8);
\end{tikzpicture}
\end{figure}
\end{frame}

\begin{frame}
\begin{figure}
\begin{tikzpicture}
\draw[line width=3] (1,1) -- (11,1);
\draw[line width=3] (1,2) -- (11,2);
\draw[line width=3] (1,3) -- (11,3);
\draw[line width=3] (1,4) -- (11,4);
\draw[line width=3] (1,5) -- (11,5);
\draw[line width=3] (1,6) -- (11,6);
\draw[line width=3] (1,7) -- (11,7);
\draw[line width=3] (1,8) -- (11,8);

\draw[line width=3] (8,1) -- (10,5);
\draw[line width=3] (8,2) -- (10,6);
\draw[line width=3] (8,3) -- (10,7);
\draw[line width=3] (8,4) -- (10,8);

\draw[line width=3] (10,1) -- (8,5);
\draw[line width=3] (10,2) -- (8,6);
\draw[line width=3] (10,3) -- (8,7);
\draw[line width=3] (10,4) -- (8,8);

\draw[line width=3] (5,1) -- (7,3);
\draw[line width=3] (5,2) -- (7,4);

\draw[line width=3] (7,1) -- (5,3);
\draw[line width=3] (7,2) -- (5,4);

\draw[line width=3] (5,5) -- (7,7);
\draw[line width=3] (5,6) -- (7,8);

\draw[line width=3] (7,5) -- (5,7);
\draw[line width=3] (7,6) -- (5,8);

\draw[line width=3] (2,1) -- (4,2);

\draw[line width=3] (4,1) -- (2,2);

\draw[line width=3] (2,3) -- (4,4);

\draw[line width=3] (4,3) -- (2,4);

\draw[line width=3] (2,5) -- (4,6);

\draw[line width=3] (4,5) -- (2,6);

\draw[line width=3] (2,7) -- (4,8);

\draw[line width=3] (4,7) -- (2,8);

\draw[red, dashed, line width=3] (1,3) -- (11,3);
\draw[red, dashed, line width=3] (10,3) -- (8,7);
\draw[red, dashed, line width=3] (1,7) -- (8,7);
\draw[red, dashed, line width=3] (5,5) -- (7,7);
\draw[red, dashed, line width=3] (1,5) -- (5,5);
\draw[red, dashed, line width=3] (2,8) -- (4,7);
\draw[red, dashed, line width=3] (1,8) -- (2,8);
\draw[red, dashed, line width=3] (2,6) -- (4,5);
\draw[red, dashed, line width=3] (1,6) -- (2,6);
\draw[red, dashed, line width=3] (5,1) -- (7,3);
\draw[red, dashed, line width=3] (1,1) -- (5,1);
\draw[red, dashed, line width=3] (2,4) -- (4,3);
\draw[red, dashed, line width=3] (1,4) -- (2,4);
\draw[red, dashed, line width=3] (2,2) -- (4,1);
\draw[red, dashed, line width=3] (1,2) -- (2,2);

\draw[red, line width=5] (1,1) circle [radius=0.1];
\draw[red, line width=5] (1,2) circle [radius=0.1];
\draw[red, line width=5] (1,3) circle [radius=0.1];
\draw[red, line width=5] (1,4) circle [radius=0.1];
\draw[red, line width=5] (1,5) circle [radius=0.1];
\draw[red, line width=5] (1,6) circle [radius=0.1];
\draw[red, line width=5] (1,7) circle [radius=0.1];
\draw[red, line width=5] (1,8) circle [radius=0.1];
\draw[red, line width=5] (2,1) circle [radius=0.1];
\draw[red, line width=5] (2,2) circle [radius=0.1];
\draw[red, line width=5] (2,3) circle [radius=0.1];
\draw[red, line width=5] (2,4) circle [radius=0.1];
\draw[red, line width=5] (2,5) circle [radius=0.1];
\draw[red, line width=5] (2,6) circle [radius=0.1];
\draw[red, line width=5] (2,7) circle [radius=0.1];
\draw[red, line width=5] (2,8) circle [radius=0.1];
\draw[red, line width=5] (4,1) circle [radius=0.1];
\draw[red, line width=5] (4,3) circle [radius=0.1];
\draw[red, line width=5] (4,5) circle [radius=0.1];
\draw[red, line width=5] (4,7) circle [radius=0.1];
\draw[red, line width=5] (5,1) circle [radius=0.1];
\draw[red, line width=5] (5,3) circle [radius=0.1];
\draw[red, line width=5] (5,5) circle [radius=0.1];
\draw[red, line width=5] (5,7) circle [radius=0.1];
\draw[red, line width=5] (7,7) circle [radius=0.1];
\draw[red, line width=5] (7,3) circle [radius=0.1];
\draw[red, line width=5] (8,7) circle [radius=0.1];
\draw[red, line width=5] (8,3) circle [radius=0.1];
\draw[red, line width=5] (10,3) circle [radius=0.1];
\draw[red, line width=5] (11,3) circle [radius=0.1];

\draw (4,1.4) node {\bf \textcolor{red}{+}};
\draw (4,3.4) node {\bf \textcolor{red}{+}};
\draw (4,5.4) node {\bf \textcolor{red}{+}};
\draw (4,7.4) node {\bf \textcolor{red}{+}};
\draw (7,3.4) node {\bf \textcolor{red}{+}};
\draw (7,7.4) node {\bf \textcolor{red}{+}};
\draw (10,3.4) node {\bf \textcolor{red}{+}};

\draw (0.3,8) node {\textcolor{red}{$f(0)$}};
\draw (0.3,7) node {\textcolor{red}{$f(4)$}};
\draw (0.3,6) node {\textcolor{red}{$f(2)$}};
\draw (0.3,5) node {\textcolor{red}{$f(6)$}};
\draw (0.3,4) node {\textcolor{red}{$f(1)$}};
\draw (0.3,3) node {\textcolor{red}{$f(5)$}};
\draw (0.3,2) node {\textcolor{red}{$f(3)$}};
\draw (0.3,1) node {\textcolor{red}{$f(7)$}};
\draw (11,3.4) node {\textcolor{red}{$F(5)$}};
\end{tikzpicture}
\end{figure}
\end{frame}

\begin{frame}
Terms from odd branches would be multiplied by $\alpha^k$ before addition.
\end{frame}

\end{document}
